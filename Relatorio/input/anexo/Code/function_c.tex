%\begin{verbatimtab}
\begin{lstlisting}[language=C, caption={function.c}, label=function-c, captionpos=b]
/*************************************************************************
FUNCTION
Author: Sergio Santos
<sergio.salazar.santos@gmail.com> 
License: GNU General Public License
Hardware: all
Date: 25102020
Comment:
Always try to make general purpose bullet proof functions !!
Very Stable
*************************************************************************/
/***Library***/
#include <avr/io.h>
#include <avr/pgmspace.h>
#include <stdarg.h>
#include <inttypes.h>
#include <math.h>
/***pc use***
#include<stdio.h>
#include<stdlib.h>
#include<string.h>
#include<errno.h>
***/
#include"function.h"
/***Constant & Macro***/
#ifndef GLOBAL_INTERRUPT_ENABLE
#define GLOBAL_INTERRUPT_ENABLE 7
#endif
#define ZERO 0
#define ONE 1
#define FUNCSTRSIZE 32
/***Global File Variable***/
char FUNCstr[FUNCSTRSIZE+ONE];
/***Header***/
unsigned int Pwr(uint8_t bs, uint8_t n);
int StringLength (const char string[]);
void Reverse(char s[]);
uint8_t  bintobcd(uint8_t bin);
/******/
unsigned int FUNCmayia(unsigned int xi, unsigned int xf, uint8_t nbits);
uint8_t FUNCpinmatch(uint8_t match, uint8_t pin, uint8_t HL);
uint8_t FUNChmerge(uint8_t X, uint8_t Y);
uint8_t FUNClmerge(uint8_t X, uint8_t Y);
uint8_t FUNChh(uint8_t xi, uint8_t xf);
uint8_t FUNCll(uint8_t xi, uint8_t xf);
uint8_t FUNClh(uint8_t xi, uint8_t xf);
uint8_t FUNChl(uint8_t xi, uint8_t xf);
uint8_t FUNCdiff(uint8_t xi, uint8_t xf);
void FUNCswap(long *px, long *py);
void FUNCcopy(char to[], char from[]);
void FUNCsqueeze(char s[], int c);
void FUNCshellsort(int v[], int n);
char* FUNCi16toa(int16_t n);
char* FUNCui16toa(uint16_t n);
char* FUNCi32toa(int32_t n);
int FUNCtrim(char s[]);
int FUNCpmax(int a1, int a2);
int FUNCgcd (int u, int v);
int FUNCstrToInt (const char string[]);
uint8_t FUNCfilter(uint8_t mask, uint8_t data);
unsigned int FUNCticks(unsigned int num);
int FUNCtwocomptoint8bit(int twoscomp);
int FUNCtwocomptoint10bit(int twoscomp);
int FUNCtwocomptointnbit(int twoscomp, uint8_t nbits);
char FUNCdec2bcd(char num);
char FUNCbcd2dec(char num);
char* FUNCresizestr(char *string, int size);
long FUNCtrimmer(long x, long in_min, long in_max, long out_min, long out_max);
unsigned char FUNCbcd2bin(unsigned char val);
unsigned char FUNCbin2bcd(unsigned val);
long FUNCgcd1(long a, long b);
uint8_t FUNCpincheck(uint8_t port, uint8_t pin);
char* FUNCprint_binary(uint8_t number);
void FUNCreverse(char* str, int len);
uint8_t FUNCintinvstr(int32_t n, char* res, uint8_t n_digit);
char* FUNCftoa(float n, char* res, uint8_t afterpoint);
/***pc use***
char* FUNCfltos(FILE* stream);
char* FUNCftos(FILE* stream);
int FUNCstrtotok(char* line,char* token[],const char* parser);
char* FUNCputstr(char* str);
int FUNCgetnum(char* x);
unsigned int FUNCgetnumv2(char* x);
int FUNCreadint(int nmin, int nmax);
***/
// uint8_t TRANupdate(struct TRAN *tr, uint8_t idata);
/***Procedure & Function***/
FUNC FUNCenable( void )
{
	uint8_t tSREG;
	tSREG=SREG;
	SREG&=~(1<<GLOBAL_INTERRUPT_ENABLE);
	// struct object
	FUNC func;
	// function pointers
	func.power=Pwr;
	func.stringlength=StringLength;
	func.reverse=Reverse;
	func.mayia=FUNCmayia;
	func.pinmatch=FUNCpinmatch;
	func.hmerge=FUNChmerge;
	func.lmerge=FUNClmerge;
	func.hh=FUNChh;
	func.ll=FUNCll;
	func.lh=FUNClh;
	func.hl=FUNChl;
	func.diff=FUNCdiff;
	func.swap=FUNCswap;
	func.copy=FUNCcopy;
	func.squeeze=FUNCsqueeze;
	func.shellsort=FUNCshellsort;
	func.i16toa=FUNCi16toa;
	func.ui16toa=FUNCui16toa;
	func.i32toa=FUNCi32toa;
	func.trim=FUNCtrim;
	func.pmax=FUNCpmax;
	func.gcd=FUNCgcd;
	func.strToInt=FUNCstrToInt;
	func.filter=FUNCfilter;
	func.ticks=FUNCticks;
	func.twocomptoint8bit=FUNCtwocomptoint8bit;
	func.twocomptoint10bit=FUNCtwocomptoint10bit;
	func.twocomptointnbit=FUNCtwocomptointnbit;
	func.dec2bcd=FUNCdec2bcd;
	func.bcd2dec=FUNCbcd2dec;
	func.resizestr=FUNCresizestr;
	func.trimmer=FUNCtrimmer;
	func.bcd2bin=FUNCbcd2bin;
	func.bin2bcd=FUNCbin2bcd;
	func.gcd1=FUNCgcd1;
	func.pincheck=FUNCpincheck;
	func.print_binary=FUNCprint_binary;
	func.ftoa=FUNCftoa;
	/***pc use***
	func.fltos=FUNCfltos;
	func.ftos=FUNCftos;
	func.strtotok=FUNCstrtotok;
	func.putstr=FUNCputstr;
	func.getnum=FUNCgetnum;
	func.getnumv2=FUNCgetnumv2;
	func.readint=FUNCreadint;
	*/
	SREG=tSREG;
	/******/
	return func;
}
// mayia
unsigned int FUNCmayia(unsigned int xi, unsigned int xf, uint8_t nbits)
{//magic formula
	unsigned int mask;
	unsigned int diff;
	unsigned int trans;
	mask=Pwr(2,nbits)-1;
	xi&=mask;
	xf&=mask;
	diff=xf^xi;
	trans=diff&xf;
	return (trans<<nbits)|diff;
}
//pinmatch
uint8_t FUNCpinmatch(uint8_t match, uint8_t pin, uint8_t HL)
{
	uint8_t result;
	result=match&pin;
	if(HL){
		if(result==match);
		else
		result=0;
	}else{
		if(result)
		result=0;
		else
		result=match;
	}
	return result;
}
// hmerge
uint8_t FUNChmerge(uint8_t X, uint8_t Y)
{
	return (X | Y);
}
// lmerge
uint8_t FUNClmerge(uint8_t X, uint8_t Y)
{
	return (X & Y);
}
// hh
uint8_t FUNChh(uint8_t xi, uint8_t xf)
{
	uint8_t i;
	i=xi&xf;
	return i;
}
// ll
uint8_t FUNCll(uint8_t xi, uint8_t xf)
{
	uint8_t i;
	i=xi|xf;
	return ~i;
}
// lh
uint8_t FUNClh(uint8_t xi, uint8_t xf)
{
	uint8_t i;
	i=xi^xf;
	i&=xf;
	return i;
}
// hl
uint8_t FUNChl(uint8_t xi, uint8_t xf)
{
	uint8_t i;
	i=xf^xi;
	i&=xi;
	return i;
}
// diff
uint8_t FUNCdiff(uint8_t xi, uint8_t xf)
{
	return xf^xi;
}
// interchange *px and *py
void FUNCswap(long *px, long *py)
{
	long temp;
	temp = *px;
	*px = *py;
	*py = temp;
}
// copy: copy 'from' into 'to'; assume to is big enough
void FUNCcopy(char to[], char from[])
{
	int i;
	i = 0;
	while ((to[i] = from[i]) != '\0')
	++i;
}
// squeeze: delete all c from s
void FUNCsqueeze(char s[], int c)
{
	int i, j;
	for (i = j = 0; s[i] != '\0'; i++)
	if (s[i] != c)
	s[j++] = s[i];
	s[j] = '\0';
}
// shellsort: sort v[0]...v[n-1] into increasing order
void FUNCshellsort(int v[], int n)
{
	int gap, i, j, temp;
	for (gap = n/2; gap > 0; gap /= 2)
	for (i = gap; i < n; i++)
	for (j=i-gap; j>=0 && v[j]>v[j+gap]; j-=gap){
		temp = v[j];
		v[j] = v[j+gap];
		v[j+gap] = temp;
	}
}
// i32toa: convert n to characters in s
char* FUNCi32toa(int32_t n)
{
	uint8_t i;
	int32_t sign;
	if ((sign = n) < 0) // record sign
	n = -n; // make n positive
	i = 0;
	do { // generate digits in reverse order
		FUNCstr[i++] = n % 10 + '0'; // get next digit
	}while ((n /= 10) > 0); // delete it
	if (sign < 0)
	FUNCstr[i++] = '-';
	FUNCstr[i] = '\0';
	Reverse(FUNCstr);
	return FUNCstr;
}
// i16toa: convert n to characters in s
char* FUNCi16toa(int16_t n)
{
	uint8_t i;
	int16_t sign;
	if ((sign = n) < 0) // record sign
	n = -n; // make n positive
	i = 0;
	do { // generate digits in reverse order
		FUNCstr[i++] = n % 10 + '0'; // get next digit
	}while ((n /= 10) > 0); // delete it
	if (sign < 0)
	FUNCstr[i++] = '-';
	FUNCstr[i] = '\0';
	Reverse(FUNCstr);
	return FUNCstr;
}
// ui16toa: convert n to characters in s
char* FUNCui16toa(uint16_t n)
{
	uint8_t i;
	i = 0;
	do { // generate digits in reverse order
		FUNCstr[i++] = n % 10 + '0'; // get next digit
	}while ((n /= 10) > 0); // delete it
	FUNCstr[i] = '\0';
	Reverse(FUNCstr);
	return FUNCstr;
}
// trim: remove trailing blanks, tabs, newlines
int FUNCtrim(char s[])
{
	int n;
	for (n = StringLength(s)-1; n >= 0; n--)
	if (s[n] != ' ' && s[n] != '\t' && s[n] != '\n')
	break;
	s[n+1] = '\0';
	return n;
}
// larger number of two
int FUNCpmax(int a1, int a2)
{
	int biggest;
	if(a1 > a2){
		biggest = a1;
	}else{
		biggest = a2;
	}
	return biggest;
}
// common divisor
int FUNCgcd (int u, int v)
{
	int temp;
	while ( v != 0 ) {
		temp = u % v;
		u = v;
		v = temp;
	}
	return u;
}
// Function to convert a string to an integer
int FUNCstrToInt (const char string[])
{
	int i, intValue, result = 0;
	for ( i = 0; string[i] >= '0' && string[i] <= '9'; ++i ){
		intValue = string[i] - '0';
		result = result * 10 + intValue;
	}
	return result;
}
// filter
uint8_t FUNCfilter(uint8_t mask, uint8_t data)
{
	uint8_t Z;
	Z=mask & data;
	return Z;
}
// ticks
unsigned int FUNCticks(unsigned int num)
{
	unsigned int count;
	for(count=0;count<num;count++)
	;
	return count;
}
// Two's Complement function
int FUNCtwocomptoint8bit(int twoscomp){
	
	//Let's see if the byte is negative
	if (twoscomp & 0B10000000){
		//Invert
		twoscomp = ~twoscomp + 1;
		twoscomp = (twoscomp & 0B11111111);
		//Cast as int and multiply by negative one
		int value = (int)(twoscomp)*(-1);
		return value;
	}
	else{
		//Byte is non-negative, therefore convert to decimal and display
		//Make sure we are never over 1279
		twoscomp = (twoscomp & 0B01111111);
		//Cast as int and return
		int value = (int)(twoscomp);
		return value;
	}
}
// Two's Complement function, shifts 10 bit binary to signed integers (-512 to 512)
int FUNCtwocomptoint10bit(int twoscomp){
	
	//Let's see if the byte is negative
	if (twoscomp & 0x200){
		//Invert
		twoscomp = ~twoscomp + 1;
		twoscomp = (twoscomp & 0x3FF);
		//Cast as int and multiply by negative one
		int value = (int)(twoscomp)*(-1);
		return value;
	}
	else{
		//Serial.println("We entered the positive loop");
		//Byte is non-negative, therefore convert to decimal and display
		twoscomp = (twoscomp & 0x1FF);
		//Cast as int and return
		//Serial.println(twoscomp);
		int value = (int)(twoscomp);
		return value;
	}
}
// Two's Complement function, nbits
int FUNCtwocomptointnbit(int twoscomp, uint8_t nbits){
	unsigned int signmask;
	unsigned int mask;
	signmask = (1<<(nbits-1));
	mask=signmask-1;
	//Let's see if the number is negative
	if (twoscomp & signmask){
		twoscomp &= mask;
		twoscomp -= signmask;
	}else{
		twoscomp &= mask;
	}
	return twoscomp;
}
// Convert Decimal to Binary Coded Decimal (BCD)
char FUNCdec2bcd(char num)
{
	return ((num/10 * 16) + (num % 10));
}
// Convert Binary Coded Decimal (BCD) to Decimal
char FUNCbcd2dec(char num)
{
	return ((num/16 * 10) + (num % 16));
}
char* FUNCresizestr(char *string, int size)
{
	int i;
	FUNCstr[size]='\0';
	for(i=0;i<size;i++){
		if(*(string+i)=='\0'){
			for(;i<size;i++){
				FUNCstr[i]=' ';
			}
			break;
		}
		FUNCstr[i]=*(string+i);
	}
	return FUNCstr;
}
long FUNCtrimmer(long x, long in_min, long in_max, long out_min, long out_max)
/***
same as arduino map function.
***/
{
	return (x - in_min) * (out_max - out_min) / (in_max - in_min) + out_min;
}
// power: raise base to n-th power; n >= 0
unsigned int Pwr(uint8_t bs, uint8_t n)
{
	unsigned int i, p;
	p = 1;
	for (i = 1; i <= n; ++i)
	p = p * bs;
	return p;
}
// Function to count the number of characters in a string
int StringLength (const char string[])
{
	int count = 0;
	while ( string[count] != '\0' )
	++count;
	return count;
}
// reverse: reverse string s in place
void Reverse(char s[])
{
	int c, i, j;
	for (i = 0, j = StringLength(s)-1; i < j; i++, j--){
		c = s[i];
		s[i] = s[j];
		s[j] = c;
	}
}
unsigned char FUNCbcd2bin(unsigned char val)
{
	return (val & 0x0f) + (val >> 4) * 10;
}
unsigned char FUNCbin2bcd(unsigned val)
{
	return ((val / 10) << 4) + val % 10;
}
long FUNCgcd1(long a, long b)
{
	long r;
	if (a < b)
	FUNCswap(&a, &b);
	if (!b){
		while ((r = a % b) != 0) {
		a = b;
		b = r;
	}
}	
return b;
}
uint8_t FUNCpincheck(uint8_t port, uint8_t pin)
{
uint8_t lh;
if(port & (1<<pin))
lh=1;
else
lh=0;
return lh;
}
char* FUNCprint_binary(uint8_t number)
{
uint8_t i,c;
for(i=128,c=0;i;i>>=1,c++){
	(number & i) ? (FUNCstr[c]='1') : (FUNCstr[c]='0');
}
FUNCstr[c]='\0';
return FUNCstr;
}
uint8_t leap_year_check(uint16_t year){
uint8_t i;
if ( !(year%400))
i=1;
else if ( !(year%100))
i=0;
else if ( !(year%4) )
i=1;
else
i=0;
return i;
}
uint8_t  bintobcd(uint8_t bin)
{
return (((bin) / 10) << 4) + ((bin) % 10);
}
/***intinvstr***/
uint8_t FUNCintinvstr(int32_t n, char* res, uint8_t n_digit)
{
uint8_t k=0;
for(res[k++] = (n % 10) + '0' ; (n/=10) > ZERO ; res[k++] = (n % 10) + '0');
for( ; k < n_digit ; res[k++] = '0');
res[k]='\0';
return k;
}
/***ftoa***/
char* FUNCftoa(float n, char* res, uint8_t afterpoint)
{
uint8_t k=ZERO;
int32_t ipart;
float fpart;
int8_t sign;
if (n < ZERO){
	n = -n;
	sign=-ONE;
}else
sign=ONE;
ipart = (int32_t) n;
fpart = n - (float)ipart;
k=FUNCintinvstr( ipart, res, ONE );
if (sign < ZERO)
res[k++] = '-';
else
res[k++] = ' ';
res[k]='\0';
Reverse(res);
if (afterpoint > ZERO) {
	res[k++] = '.';
	FUNCintinvstr( fpart * pow(10, afterpoint), res+k, afterpoint );
	Reverse(res+k);
}	
return res;
}
/******
int gcd( int a, int b ) {
int result ;
// Compute Greatest Common Divisor using Euclid's Algorithm
__asm__ __volatile__ ( "movl %1, %%eax;"
"movl %2, %%ebx;"
"CONTD: cmpl $0, %%ebx;"
"je DONE;"
"xorl %%edx, %%edx;"
"idivl %%ebx;"
"movl %%ebx, %%eax;"
"movl %%edx, %%ebx;"
"jmp CONTD;"
"DONE: movl %%eax, %0;" : "=g" (result) : "g" (a), "g" (b)
);
return result ;
}
//
float sinx( float degree ) {
float result, two_right_angles = 180.0f ;
// Convert angle from degrees to radians and then calculate sin value
__asm__ __volatile__ ( "fld %1;"
"fld %2;"
"fldpi;"
"fmul;"
"fdiv;"
"fsin;"
"fstp %0;" : "=g" (result) : 
"g"(two_right_angles), "g" (degree)
) ;
return result ;
}
//
float cosx( float degree ) {
float result, two_right_angles = 180.0f, radians ;
// Convert angle from degrees to radians and then calculate cos value
__asm__ __volatile__ ( "fld %1;"
"fld %2;"
"fldpi;"
"fmul;"
"fdiv;"
"fstp %0;" : "=g" (radians) : 
"g"(two_right_angles), "g" (degree)
) ;
__asm__ __volatile__ ( "fld %1;"
"fcos;"
"fstp %0;" : "=g" (result) : "g" (radians)
) ;
return result ;
}
//
float square_root( float val ) {
float result ;
__asm__ __volatile__ ( "fld %1;"
"fsqrt;"
"fstp %0;" : "=g" (result) : "g" (val)
) ;
return result ;
}
*/
/***pc use***
char* FUNCfltos(FILE* stream)
{
int i, block, NBytes;
char caracter;
char* value=NULL;
for(i=0, block=4, NBytes=0; (caracter=getc(stream)) != EOF; i++){
	if(i==NBytes){
		NBytes+=block;
		value=(char*)realloc(value, NBytes*sizeof(char));
		if(value==NULL){
			perror("fltos at calloc");
			break;
		}
	}
	*(value+i)=caracter;
	if(caracter=='\n'){
		*(value+i)='\0';
		break;
	}
}
return value;
}
char* FUNCftos(FILE* stream)
{
int i, block, NBytes;
char caracter;
char* value=NULL;
for(i=0, block=8, NBytes=0; (caracter=getc(stream)) != EOF; i++){
	if(i==NBytes){
		NBytes+=block;
		value=(char*)realloc(value, NBytes*sizeof(char));
		if(value==NULL){
			perror("ftos at calloc");
			break;
		}
	}
	*(value+i)=caracter;
}
return value;
}
int FUNCstrtotok(char* line,char* token[],const char* parser)
{
int i;
char *str;
str=(char*)calloc(strlen(line),sizeof(char));
for (i = 0, strcpy(str,line); ; i++, str = NULL) {
	token[i] = strtok(str, parser);
	if (token[i] == NULL)
	break;
	printf("%d: %s\n", i, token[i]);
}
free(str);
return i;
}
char* FUNCputstr(char* str)
{
int i; char* ptr;
ptr = (char*)calloc(strlen(str), sizeof(char));
if(ptr == NULL){
	perror("NULL!\n");
	return NULL;
}
for(i=0; (ptr[i] = str[i]); i++){
	if(ptr[i] == '\0')
	break;
}
return (ptr);
}
int FUNCgetnum(char* x)
{
int num;
if(!sscanf(x, "%d", &num))
num=0;
return num;
}
unsigned int FUNCgetnumv2(char* x)
{
unsigned int RETURN;
unsigned int value;
unsigned int n;
errno=0;
n=sscanf(x,"%d",&value);
if(n==1){
	RETURN=value;
}else if(errno != 0){
	perror("getnum at sscanf");
	RETURN=0;
}else{
	RETURN=0;
}
return RETURN;
}
int FUNCreadint(int nmin, int nmax)
{
int num;
int flag;
for(flag=1; flag;){
	for( num=0; !scanf("%d",&num);getchar())
	;
	//printf("num: %d nmin: %d nmax: %d\n",num, nmin, nmax);
	if((num < nmin) || (num > nmax))
	continue;
	flag=0;
}
return num;
}
***/
/***Interrupt***/
/***Comment***
*************/
/***EOF***/
\end{lstlisting}
%\end{verbatimtab}
%%%%%%%%%%%%%%%%%%%%%%%%%%%%%%%%%%%%%%%%%%%%%%%%%%%%%%%%%%%%%%%%
