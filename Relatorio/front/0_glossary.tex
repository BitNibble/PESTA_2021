%---------------------------------------------------------
%	GLOSSARY
%---------------------------------------------------------
% Only the used entries will be displayed in the printed list, ie, you need to used a term at least once
% In italic if not in the main document language
% terms definition usage:
% \newglossaryentry{<tag>}{name={<term>},description={<description of the term>}}
%\makeglossaries
\newglossaryentry{gloss}{
name={glossário}, 
description={é uma lista alfabética de termos de um determinado domínio de conhecimento com a definição desses mesmos termos.},
}
\newglossaryentry{pack}{
name={\textit{package}}, 
description={é um ficheiro ou conjunto de ficheiros que contêm comandos \LaTeX{} extra que adicionam novas funcionalidades de estilo ou modificam aquelas já existentes.},
sort={package}	%needed for sorting when using LaTeX commands in the 'name' field
}
\newglossaryentry{lipsum}{
name={\textit{Lorem Ipsum}}, 
description={é uma sequência de palavras, geralmente latinas, utilizada para preencher o espaço destinado a texto numa publicação, por forma a testar as opções de formatação e edição e o arranjo dos elementos gráficos antes da inserção do conteúdo.},
sort={Lorem Ipsum}
}
%%%%%%%%%%%%%%%%%%%%%%%%%%%%%%%%%%%%%%%%%%%%%%%%%%%%%%%%%%%%%%%%
\newglossarystyle{mylong}{%
	\setglossarystyle{long}%
	\renewenvironment{theglossary}%
	{\begin{longtable}[l]{@{}p{\dimexpr 3cm-\tabcolsep}p{.7\hsize}}}% <-- change the value here
		{\end{longtable}}%
}
\newglossaryentry{latex}
{
	name=latex,
	description={Is a mark up language specially suited 
		for scientific documents}
}
\newglossaryentry{maths}
{
	name=mathematics,
	description={Mathematics is what mathematicians do}
}
\newglossaryentry{formula}{
	name=formula,
	description={A mathematical expression}
}
\newglossaryentry{Biofouling}{
	name=Biofouling,description={Some description}
}
\newglossaryentry{symb:Pi}{
	name=\ensuremath{\pi},
	description={Geometrical value}
}
%%%%%%%%%%%%%%%%%%%%%%%%%%%%%%%%%%%%%%%%%%%%%%%%%%%%%%%%%%
